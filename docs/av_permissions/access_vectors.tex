\documentclass{article}
\title{Meaning of permissions in SELinux(Ver 1)}
\author{Yuichi Nakamura \thanks{The George Washington University,
ynakam@gwu.edu}}
\begin{document}
\def\labelenumi{(\theenumi)}
\maketitle
\tableofcontents
\newpage

\section{Introduction}
\subsection{About this document}
Meaning of SELinux's permissions(operations that are restricted by permissions) are
analyzed and shown in this document.
For feedback, please send e-mail to ynakam@gwu.edu.

\subsection{Terminology and notation}
\begin{itemize}
 \item Notation\\
       read(2) means, you should refer to Linux man pages. This
example means {\it man 2 read}.
 \item Access vector permission\\
       The term {\it Access vector permission} means, permission defined
       in SELinux(such as read,write,send\_msg). {\it Access vector
       permission } is often called simply {\it permission } in this
       document. The usage of the term is from \cite{configpolicy}.
\end{itemize} 

\subsection{Motivation}
The design SELinux Policy Editor\cite{seedit} is based on non-LSM based SELinux released at the time of Jan 2003. After that SELinux is re-implemented
using LSM. As a result, meaning of access vector permissions had been changed, and many
permissions are added. Before re-designing SELinux Policy Editor, we have to understand the meaning of permissions.  However, the meaning of them
is not well documented. \cite{selinuximplementation} is a good
documentation of implementing SELinux, but the meaning of access
vectors are not fully covered. \cite{tresysav} is a quick reference of
permissions, but the description is short.  Therefore, I decided
to analyze the meaning of all permissions.

\section{Analysis method}
The version of SELinux used is that in Linux kernel 2.6.13. Analysis of permissions are based on source code analysis
of Linux 2.6.13.  The process is following.
\begin{enumerate}
 \item  Find value corresponding to the permission from
	security/selinux/include/ av\_permissions.h\\
	In av\_permissions.h, permission is defined as a constant
	value. For example, when we want to analyze permission {\it
	read} for object class {\it file}. 
	We can find following in av\_permissions.h.
	\begin{verbatim}
	#define FILE__READ     0x00000002UL		
	\end{verbatim}
 \item  Analyze how the constant is used.\\
	In the example above, we analyze how FILE\_\_READ is used in
	source code. And find out how the permission is checked. 
	lxr \cite{lxr} is useful. 
\end{enumerate}
In some cases above process is not enough. 
\begin{itemize}
 \item Object class capability\\
       For permissions in object class capability, constants
       CAPABILITY\_\_* are defined in av\_permissions.h. However, we can
       not find such constants in source code. We tend to think those
       are not checked, but they are
       actually checked. They are checked in {\it capable} Linux
       function. Let's see permission {\it cap\_sys\_admin} as an
       example. The permission is checked in the statement {\it
       capable(CAP\_SYS\_ADMIN)}, then {\it selinux\_capable} and
       {\it task\_has\_capability} are called, and permission 
       {\it cap\_sys\_admin} is checked. In the check, values defined in
       av\_permissions.h do not appear explicitly.
 \item Object classes related to files and sockets\\
       In these object classes, some permissions are inherited from
       object class {\it file}. 
       We have to pay attention to analyze them. Such permissions
       use value FILE\_\_*.  
       For example, when we analyze permission {\it read} in object
       class {\it tcp\_socket}, {\it read } permission is inherited
       from {\it file}. We can find TCP\_SOCKET\_\_READ in
       av\_permissions.h. However, we have to also analyze the behavior
       of FILE\_\_READ. 
\end{itemize}

\section{Meaning of permissions}
\subsection{permissions related to files}
In the following subsection, operations restricted by permissions are
described.
\subsubsection{Object classes}
Object classes related to file are summarized in Table \ref{tab:obfile}.
\begin{table}[h]
\caption{Object classes related to file}\label{tab:obfile}
\begin{center}
\begin{tabular}{|l|l|}
\hline
Object class & For what kind of file?\\
\hline
 file & Normal file\\
\hline
 blk\_file &  Block device file\\
\hline
 chr\_file & Character device file\\
\hline
fifo\_file & Special file for FIFO\\
\hline
lnk\_file & Symbolic link\\
\hline
sock\_file & Special file for Unix domain socket\\
\hline
dir & Directory\\
\hline
\end{tabular}
\end{center}
\end{table}

\subsubsection{permissions common to object classes related to file}
\begin{itemize}
 \item ioctl\\
       Control attribute of device. It is checked in ioctl(2).
 \item read\\
       Read file. It is checked in read(2).
 \item  write\\
       Write to file. It is checked in write(2).
 \item create \\
       Open and create new file, directory and symbolic link.        
 \item  getattr\\
       Get file attribute (such as last modified). It is checked in stat(2).
 \item setattr\\
       Modify file attribute.  It is checked in kernel functions that
       changes file attribute.
 \item  lock\\
       Lock file. It is checked in flock(2) and fcntl(2)
 \item  relabelfrom, relabelto\\
       Relabel file. When domain A relabel file whose type is B to type
       C. A must have relabelfrom to B and relabelto to C.
 \item append\\
       Append to file. It is checked when opening file as append mode.
 \item  unlink\\
       Delete file. It is checked in  unlink(2).
 \item  link\\
       Create hard link. When domain A want to create hard link for file
       whose type is B, A must have link permission to B.
 \item rename\\
       Rename file. It is checked in rename(2). rename(2) is used in
       such as mv command.
 \item execute\\
       Execute file with domain transition. Link shared library.
 \item  swapon\\ 
       It is not used. It is defined in SELinux source as
       FILE\_\_SWAPON but not used. Originally, it controlled swapon system
       call, but this was dropped when merged into mainline Linux
       kernel. For domain A to do swapon system call successfully, A need {\it getattr
       read and write } permissions. So without {\it swapon} permission, swapon system call can be restricted.
% $B;H$o$l$F$J$$!)!)(B
%       mkswap$B%3%^%s%I(B, swapon$B$K$h$j!$(Bswap$B$O!"IaDL$N%U%!%$%k$K:n@.$9$k$3$H$,$G$-(B
%       $B$?!#$G!$(Bswapon$B$9$k$H!$%+!<%M%k$+$i%9%o%C%W$,H/@8$7$?$H$-!"%U%!%$(B
       %$B%k$,GK2u!#FC$K%"%/%;%9%Y%/%?$O$A$'$C$/$5$l$:!#(B
%       -> seedit$B$G$O!"(Bblk_file$B$N$_!#(B
% swapon$B$O(Bcap_admin$B$G$b@)8f$5$l$k(B
% mkswap $B;~E@$G!$(Bread,write$B$N%A%'%C%/$,$5$l$k!#(Bswapon$B;~$b(Bread write$B%A%'%C(B
       %$B%/$5$l$k!#(Bgetattr$B$,$J$$$H(Bswapon$B$G$-$J$$!#(B
       
 \item quotaon\\
       Enable quota to disk device file. It is checked in quotactl(2)(Q\_QUOTAON flag).
 \item mounton\\
       %  xxxx may be important!!xxxx\\
       Use directory as a mount point. It is checked in mount(2).       
\end{itemize}
      
\subsubsection{permissions specfic to object class {\it file,
   blk\_file}}
   \begin{itemize}
    \item execute\_no\_trans\\
	     Execute file without domain transition.
    \item entrypoint\\
	     Use file as a entry point for domain transition.
    \item  execmod\\
	  Attempt to load executable in specific condition. The condition
	  is quoted from \cite{selinuximplementation} below. {\it It first checks whether the caller is attempting to make executable a file mapping that has had some
	  copy-on-write done, indicating that it may include modified content. If so, then the hook function
	  performs a file execmod permission check.}
   \end{itemize}
    
\subsubsection{permissions specific to dir}
\begin{itemize}
 \item  add\_name\\
	Add entry to directory. It is checked in rename(2) and link(2).
 \item  remove\_name\\
	Remove entry from directory. It is checked in unlink(2).
 \item  reparent\\
	Change parent directory. It is checked in rename(2).
 \item search\\
	When opening file and directory or changing directory, {\it search }
	permission is checked.  {\it search } is checked to all ancestor
       directories. For example, when {\it cd /etc/selinux/seedit/}
       command(suppose the domain is foo\_t) is run, {\it search } is
       checked to /, /etc, /etc/selinux and /etc/selinux/seedit.
 \item rmdir\\
	Remove directory. It is checked in rmdir(2).
\end{itemize}

\subsection{permissions related to sockets}
\subsubsection{Object Classes}
In SELinux,object classes are related to sockets. SELinux categorizes
sockets by protocol family and type. protocol family is {\it domain} and
type is {\it type } in socket system call \footnote{See man socket(2)}.
Table \ref{tab:objsock} shows relationship between object class,
protocol family and type. 
{\tiny
\begin{table}[h]
\caption{Object classes related to socket, partly quoted from \cite{selinuximplementation}}\label{tab:objsock}
\begin{center}
\begin{tabular}{|l|l|l|}
%\begin{tabular}{|p{10em}|p{6em}|p{6em}|p{10em}|}
\hline
Object class & Protocol Family & Type \\
\hline
tcp\_socket & PF\_INET, PF\_INET6 &SOCK\_STREAM  \\
\hline
udp\_socket & PF\_INET, PF\_INET6 &SOCK\_DGRAM \\
\hline
rawip\_socket &  PF\_INET, PF\_INET6&SOCK\_RAW\\
\hline
 unix\_stream\_socket&PF\_UNIX & SOCK\_STREAM\\
\hline
 unix\_dgram\_socket&PF\_UNIX & SOCK\_DGRAM\\
\hline
 packet\_socket&PF\_PACKET & all \\
\hline
key\_socket & PF\_KEY & all  \\
\hline
netlink\_route\_socket & PF\_NETLINK & NETLINK\_ROUTE \\
\cline{1-1}\cline{3-3}
 netlink\_firewall\_socket&  &NETLINK\_FIREWALL  \\
\cline{1-1}\cline{3-3}
netlink\_tcpdiag\_socket & & NETLINK\_TCPDIAG \\
\cline{1-1}\cline{3-3}
netlink\_nflog\_socket& & NETLINK\_NFLOG  \\
\cline{1-1}\cline{3-3}
netlink\_xfrm\_socket& &  NETLINK\_XFRM  \\
\cline{1-1}\cline{3-3}
netlink\_selinux\_socket& & NETLINK\_SELINUX  \\
\cline{1-1}\cline{3-3}
netlink\_audit\_socket& & NETLINK\_AUDIT  \\
\cline{1-1}\cline{3-3}
netlink\_ip6fw\_socket& &  NETLINK\_IP6\_FW  \\
\cline{1-1}\cline{3-3}
netlink\_dnrt\_socket & & NETLINK\_DNRTMSG  \\
\cline{1-1}\cline{3-3}
netlink\_kobject\_uevent \_socket& & NETLINK\_KOBJECT\_UEVENT \\
\cline{1-1}\cline{3-3}
netlink\_socket& &ALL other types  \\
\hline
socket & \multicolumn{2}{|l|}{all sockets unmatched above}\\ 
\hline
\end{tabular}
\end{center}
\end{table}
}

The short description of each sockets are below.
\begin{itemize}
 \item tcp\_socket, udp\_socket\\
      These are trivial, TCP and UDP socket. 
 \item rawip\_socket, packet\_socket\\
       These are related to socket to send raw packets. These can be used
       by attacker to create fake packet. 
 \item  unix\_stream\_socket\& unix\_dgram\_socket\\
       These are unix domain socket, socket to communicate with
       processes in the same machine.
 \item netlink*socket\\
       These are related to netlink socket. Netlink socket is a socket
       to communicate with kernel.
 \item key\_socket\\
       This is a socket used for IPSEC.
 \item socket\\
       Sockets that does not match all of above. From socket(2), 
       unmatched sockets will be those whose protocol family are PF\_IPX(IPX-Novell protocols), PF\_X25(ITU-T X.25
       /ISO-8208 protocol), PF\_AX25(Amateur radio AX.25 protocol),
       PF\_ATMPVC(Access to raw ATM PVCs) and PF\_APPLETAL(Appletalk).   
\end{itemize}

\subsubsection{permissions common to sockets}\label{sec:commonsock}
\begin{itemize}
 \item relabelfrom, relabelto\\
       These permission is defined, bot not used.
       %$B>-MhI|3h$N2DG=@-$,$"$k(B

 \item Target type is domain who created socket \\
       For following , target type is domain who created socket. 
 \begin{itemize}
  \item   read\\
	  Read data from socket. This is checked in system call
	  recvmsg. In other system calls related to socket read(such as recvfrom), recvmsg
	  call are internally used.
  \item  write\\
	  Write data to socket. This is checked in system call
	  sendmsg. sendmsg is internally used in other system calls
	  related to socket write.
  \item  create\\
	  Create socket. This is checked when socket is created.
  \item  getattr\\
	  Get name of socket by  getsockname and getpeername system call.
	  
  \item  bind\\
	  Usage of bind system call. bind system call is to give name to
	  socket.
  \item  connect\\
	  Usage of connect system call. connect system call is used to
	  initiate network connection
  \item  listen, accept\\
	  Usage of listen and accept system call. These calls are used to
	  wait network connection.
  \item  getopt\\
	  Get socket option by getsockopt system call.
  \item  setopt\\
	  Set socket option by setsockopt system call.
  \item  shutdown\\
	  Terminate connection by shutdown system call.
  \item ioctl\\
	  Set and get attribute of socket by ioctl system call.
  \item append\\
	  open socket with O\_APPEND option,but it does not make sense
	  for socket.
  \item lock\\
	  Lock file descriptor for socket by flock and fcntl system call.
  \item setattr\\
	  Set inode attribute of socket.
 \end{itemize}

       As an example, let's see when a\_t domain communicate b\_t
       domain by TCP. 
       For convenience of explanation, we see
       only create and write permissions. 
       \begin{enumerate}
	\item a\_t domain process open TCP socket, and establish connection with
	      process whose domain is b\_t\\
	      TCP socket is created, and {\it create} permission is
	      checked. The created socket has type a\_t.
	      {\it allow a\_t a\_t:tcp\_socket create;} is necessary to
	      allow it.
	\item a\_t domain write data to socket.\\
	      {\it write } permission is checked.
	      {\it allow a\_t a\_t:tcp\_socket write;} is necessary to
	      allow this operation. Note that target type is {\it not}
	      b\_t, because type of socket which a\_t is writing is {\it
	      a\_t}.	      
       \end{enumerate}
       
 \item Target type is domain of peer socket\\
       \begin{itemize}
	\item sendto\\
	      Connect by connect call and send data by sendmsg
	      call. Note that this permission is checked only in unix
	      data gram socket.
	      % unix_dgram_connect,  unix_dgram_sendmsg$B$N%A%'%C%/$,$5$l(B
	      %$B$F$?(B
       \end{itemize}
       The target type of above permission is domain of peer socket. 
       For example, when domain a\_t want to send data to domain b\_t,
       permission check is done using {\it domain:a\_t, type:b\_t,
       objectclass:unix\_dgram\_socket, permission:sendto }. This
       means communication between domains are checked. To allow this       
	\begin{verbatim}
       allow a_t b_t:unix_dgram_socket sendto;			
	\end{verbatim}
       must be described in policy.         
 \item Target type is port\\
       In following permissions, type of port is used as target
       type. In SELinux, port numbers are labeled.

       \begin{itemize}
	\item  name\_bind\\
	       Open port.
	\item recv\_msg\\
	       Receive data from port. This is checked in kernel
	       function processing incomming data. Target type is type of source port.
	\item send\_msg\\
	       Send data to port. Target type is type of destination port.
       \end{itemize}       
       For example,
       When a\_t domain want to receive data from peer whose port is tcp 80(assuming type is http\_port\_t),
       \begin{verbatim}
		allow a_t httpd_port_t:tcp_socket recv_msg;
       \end{verbatim}
       must be specified.
\end{itemize}


%% $B$3$l$O!$%?!<%2%C%H$,%]!<%H$@$C$?$j$9$k!#8e$GD4::!#(B      
%%$B$3$l$b!#%?!<%2%C%H$,%]!<%H$+$b(B 

%%$B0J2<$N%"%/%;%9%Y%/%?$N(Btarget$B%?%$%W$OA4It!V(Bself$B!J%I%a%$%s$HF1$8!K!W$K$J(B
%       $B$k!#(Btarget$B$O%=%1%C%H$N(Binode$B$N%?%$%W$K$J$k$,!$%=%1%C%H$N(Binode$B$N%?(B
%        $B%$%W$O%I%a%$%s$HF1$8$K$J$k!#(B{\it If the labeling behavior is to inherit the inode SID directly from the allocating task, then the function
%copies the task SID from the inode security structure into its own SID field. This behavior is used for
%private objects such as socket and pipes. (p34, 14.1.3)}
%$B%=%1%C%H$K$b(Bsid$B$rIUM?$G$-$?$,!$$$$^$O(BUNIX$B%I%a%$%s$K$7$+IUM?$G$-$J$$(B
      
%%$B6&DL$@$,!"<B$O(Bunix dgram$BFCM-(B
%%socket_unix_may_send(struct socket *sock,
%%                                        struct socket *other)
%%$B$3$l$O!$%=!<%9!"%?!<%2%C%H$,$A$c$s$H@)8f2D(B Unix dgram$B%=%1%C%H$G%A%'%C(B


\subsubsection{Object class netlink\_socket, packet\_socket, key\_socket
,unix\_dgram\_socket }
For these object classes, all permissions are the same as those
specified in section \ref{sec:commonsock}.


\subsubsection{Object class  unix\_stream\_socket }
In addition to permissions in \ref{sec:commonsock}, following are
defined. These permissions are checked between subject domain and
domain of peer. 
\begin{itemize}
 \item  connectto\\
	Connect to peer by connect system call
 \item  newconn\\
	This is not used. Defined as UNIX\_STREAM\_SOCKET\_\_NEWCONN,
	but not actually used.
 \item acceptfrom\\
	This is not used. Defined as UNIX\_STREAM\_SOCKET\_\_ACCEPTFROMN,
	but not actually used.
\end{itemize}
    

\subsubsection{Object class  tcp\_socket }
In addition to permissions in \ref{sec:commonsock}, following are
defined. 
\begin{itemize}
 \item    connectto  \\
	  Defined as TCP\_SOCKET\_\_CONNECTTO, but not used.
 \item  newconn\\
	  Defined as TCP\_SOCKET\_\_NEWCONN but not used.
 \item  acceptfrom\\
	  Defined as TCP\_SOCKET\_\_ACCEPTFROM but not used.
	  
 \item  node\_bind \\ 
	  Name socket by bind system call. Target type is type of node(Network address).
	  %err = avc_has_perm(isec->sid, sid, isec->sclass(Node$B$N%/%i%9(B), node_perm, &ad);

 \item  name\_connect
	  Begin network connetion by connect system call. Target type is
	  port number.
   % err = avc_has_perm(isec->sid, sid(Port), isec->sclass, TCP_SOCKET__NAME_CONNECT, &ad);
\end{itemize}

\subsubsection{Object class udp\_socket, rawip\_socket }
In addition to permissions in \ref{sec:commonsock}, node\_bind is  defined.
The meaning of node\_bind is the same as that of tcp\_socket.

\subsubsection{Object class netlink\_nflog\_socket, netlink\_selinux\_socket and netlink\_dnrt\_socket}
permissions are the same as \ref{sec:commonsock}.

\subsubsection{Object class netlink\_audit\_socket}
In addition to permissions in \ref{sec:commonsock}, following permissions are defined. These permissions are checked when sending
message to CAPP(Controlled Access Protection Profile)\cite{capp} audit system.
\begin{itemize}
 \item  nlmsg\_read\\
	Send message  to query the status of LauS.
 \item  nlmsg\_write\\
	Send message to change configuration of LauS.
 \item  nlmsg\_relay\\
        Send user space log  message to LauS. 
 \item nlmsg\_readpriv\\
	Send message to obtain configuration of LauS.
\end{itemize}


%%%%$B7k6I$I$s$J>pJsN.$,@)8f$G$-$k$+$r$^$H$a$kI,MW$,$"$k!#(B
%%%xxxxxxxxxnetlink $B$^$@(B
\subsubsection{netlink\_route\_socket}
netlink\_route\_socket is used to restrict access to netlink socket that
is used to configure kernel routing table. In addition to permissions in \ref{sec:commonsock}, following permissions are defined.
%man rtnetlink
\begin{itemize}
 \item nlmsg\_read\\
       Send messsage to request to read kernel routing table.
 \item nlmsg\_write\\
       Send message to request to write kernel routing table.
\end{itemize}

\subsubsection{Object class netlink\_firewall\_socket }

This object class is to control access to IPv4 firewall. 
%netlink(7) NETLINK_FIREWALL
 In addition to permissions in \ref{sec:commonsock}, following permissions are defined. 
\begin{itemize}
 \item         nlmsg\_read \\ 
	       This is defined but not used.
 \item         nlmsg\_write\\
	       Send message whose mode is IPQM\_VERDICT or IPQM\_MODE
	       defined in ip\_queue.h. 
\end{itemize} 

\subsubsection{Object class netlink\_tcpdiag\_socket}
netlink\_tcpdiag\_socket is used to restrict usage of  netlink socket
for network monitoring kernel module enabled by CONFIG\_IP\_TCPDIAG
kernel compile option.

% 11 #define NETLINK_INET_DIAG       4       /* INET socket monitoring                       */
% http://www.linux-m32r.org/lxr/http/source/net/ipv4/inet_diag.c


In addition to permissions in \ref{sec:commonsock}, following permissions are defined. 
\begin{itemize}
 \item    nlmsg\_read\\
	  Send message requesting to get infomation about TCP and
	  DCCP protocol. 
	  %TCPDIAG_GETSOCK DCCPDIAG_GETSOCK
 \item nlmsg\_write\\
 This is defined but not used.
\end{itemize}

\subsubsection{netlink\_xfrm\_socket} % IPSEC
netlink\_tcpdiag\_socket is used to restrict usage of
netlink\_xfrm\_socket to configure IPSEC.In addition to permissions
in \ref{sec:commonsock}, following permissions are defined. 
\begin{itemize}
 \item    nlmsg\_read\\
	  Send message to request to read  IPSEC parameter.
 \item nlmsg\_write\\
	  Send message to request to set IPSEC parameter.
\end{itemize}


\subsubsection {Object class netlink\_ip6fw\_socket}
This object class is defined, but not used.

           
\subsection{permissions related to other network elements}
\subsubsection{Object class netif}
Following permissions are defined. In these, target type is network
interface\footnote{SELinux labels network interface.}.
\begin{itemize}
 \item  tcp\_recv\\
	This is checked when tcp socket receives data from network
	interface.\\   
 \item   tcp\_send\\
	This is checked when tcp socket sends data to network
	interface.\\   
 \item udp\_recv\\
	This is checked when udp socket receives data from network
	interface.\\  
 \item udp\_send\\
	This is checked when udp socket sends data from network
	interface.\\  
 \item rawip\_recv\\
	This is checked when raw socket(RAW socket and packet socket) receives data from network
	interface.\\  
 \item rawip\_send\\
	This is checked when raw socket sends data from network
	interface.\\  
\end{itemize}


\subsubsection{Object class node}
Following permissions are defined. In these, target type is network
node(IP address).
\begin{itemize}
 \item     tcp\_recv,        tcp\_send, udp\_recv,udp\_send,
	   rawip\_recv,        rawip\_send\\
	   The same as those in class netif except target type is type
	   of node.
 \item enforce\_dest\\
	   Defined as NODE\_\_ENFORCE\_DEST but not used.
\end{itemize}
   

\subsection{permissions related to IPC}
\subsubsection{Object classes}
\begin{itemize}
 \item ipc \\
       Defined SECCLASS\_IPC,but not used. 
 \item msgq\\
       IPC message queue. SELinux labels  msgq. The type is the
       same as doamin of creating process.
 \item sem\\
       IPC semaphore.SELinux labels semaphore msgq. The type is the
       same as doamin of creating process.
 \item shm\\
       IPC shared memory. SELinux labels shared memory. The type is the
       same as domain of creating process.
 \item msg\\
       Message used in message queue. SELinux labels message. The type
       is the same as message queue to which a process is going to send
       to msgq.
\end{itemize}

\subsubsection{permissions common to all ipc object classes}
\begin{itemize}
 \item create\\
       Create IPC object.
 \item destroy\\
       Destroy IPC object by shmctl(option IPC\_RMID ).
 \item getattr\\
       Get information about IPC by shmctl, msgctl and semctl (option IPC\_STAT) 
 \item setattr\\
       Change attributie of IPC object by shmctl,msgctl and
       semctl(option IPC\_SET)
 \item read\\
       Meaning of this is different depending on object class.
       \begin{itemize}
	\item shm\\
	      Attach shared memory to process by using shmat SHM\_RDONLY option.
	\item msgq\\
	      Read message from message queue.
	\item sem\\
	      Get value of semaphore by semctl(GETALL option) and semop.
       \end{itemize}
       
 \item write\\
       Meaning of this is different depending on object class.
       \begin{itemize}
	\item shm\\
	      Attach shared memory to process by shmat not SHM\_RDONLY option.
	\item msgq\\
	      Send message to message queue.
	\item sem\\
	      Change value of semaphore by semctl(SETALL option) and semop.
       \end{itemize}
       
 \item associate\\
       \begin{itemize}
	\item sem\\
	      In addition to operations restricted by getattr, get id by
	      semget.
	\item shm\\
	      In addition to operations restricted by getattr, get id by shmget
	\item msgq\\
	      Get id by msgget
       \end{itemize}

 \item unix\_read\\
       Operations that read ipc object. This is checked when ipcperms
       kernel function(with S\_IRUGO flag) is called. ipcperms function
       with S\_IRUGO flag is called when ipc object is read.       
 \item unix\_write\\
       Operations that write or modify ipc object. This is checked when
       ipcperms kernel function(with S\_IWUGO flag) is called. ipcperms function
       with S\_IWUGO flag is called when ipc object is written or modified.       
\end{itemize}

\subsubsection{Object class msgq}
In addition to permissions common to IPC, enqueue is defined.
\begin{itemize}
 \item         enqueue\\
	       This is the same as write.
\end{itemize}

\subsubsection{Object class msg}
There are only two permissions in msg. Object classes common to IPC
    are not used.
\begin{itemize}
 \item    send\\
	  This is the same as write of msgq, except that target type is
	  type of message.
 \item    receive\\
	  This is the same as read of msgq, except that target type is
	  type of message.
\end{itemize}
As a target type type of message is used above. However, by default,
type of message is the same as type of msgq. So, above permissions
are same as write and read for msgq.

\subsubsection{Object class sem}
permissions are the same as those common to IPC.

\subsubsection{Object class shm}
In addition to permissions common to IPC, lock is defined.
\begin{itemize}
 \item         lock\\
	       Lock shared memory by shmctl with SHM\_LOCK or
	       SHM\_UNLOCK option.	       
\end{itemize}


\subsection{Object class capability}
 \begin{itemize}
 \item    chown\\
	  Change owner of file by chown. 
 \item dac\_override\\
	  Skip ordinary Linux's permission check(DAC). 

 \item  dac\_read\_search\\
	  Skip ordinary Linux's permission check about read and
	  directory search. 
 \item   fowner
	  \begin{itemize}
	   \item  Skip permission check in chmod and utime\\
	   \item Change acl(Posix ACL)\\
		  %$B$3$l$i$O(B,setattr$B$H=E$J$k(B
	  \end{itemize}
  \item       fsetid\\
	  Some operations related to setuid.Quoted from capabilities(7):	  
	  {\it Don't  clear  set-user-ID  and  set-group-ID bits when a file is
              modified; permit setting of the  set-group-ID  bit  for  a  file
              whose  GID  does not match the file system or any of the supple-
              mentary GIDs of the calling process.}

	  
  \item  kill\\
	  Skip permission check about kill. The same as CAP\_KILL
	  
  \item setgid\\
	  Change GID for process and socket. Quoted from
	  capabilities(7):{\it 
     Allow  arbitrary manipulations of process GIDs and supplementary
              GID list; allow forged GID when passing socket  credentials  via
              Unix domain sockets.}

  \item setuid\\
	  Change UID for process and socket. The same as CAP\_SETUID.
	  Quoted from capabilities(7):{\it Allow arbitrary manipulations of process UIDs (setuid(2), etc.);
	  allow forged UID when passing socket credentials via Unix domain
	  sockets.}
  \item  setpcap\\ 
	  Change capability. The same as CAP\_SETPCAP. Quoted from capabilities(7):{\it
	  Grant  or  remove any capability in the caller's permitted capa-
	  bility set to or from any other process.}
  \item linux\_immutable \\ 
	  Set immulable flag on files that support immutable flag.  The
	  same as CAP\_LINUX\_IMMUTABLE.
	  
  \item net\_bind\_service\\
	  Bind well known port.    The same as CAP\_NET\_BIND\_SERVICE.

  \item net\_broadcast\\
	  Not used.
	  
  \item net\_admin\\
	  The same as  CAP\_NET\_ADMIN. Quoted from capabilities(7):{\it
	  Allow various network-related operations (e.g.,  setting  privi-
	  leged  socket options, enabling multicasting, interface configu-
	  ration, modifying routing tables). }
	  
  \item net\_raw\\
	  Use raw and packet sockets. 
  \item ipc\_lock\\
	  Memory lock using mlock, mlockall, shmctl. The same as
	  CAP\_IPC\_LOCK.
	  
  \item ipc\_owner\\
	 Skip permision check about IPC. The same as  CAP\_IPC\_OWNER.
	  
  \item sys\_module\\	  
	  Load and unload kernel module. The same as CAP\_SYS\_MODULE.
  \item sys\_rawio\\
	  Manipulate I/O port by iopl and ioperm. Access /proc/kcore.
	  The same as CAP\_SYS\_RAWIO.
  \item         sys\_chroot\\
	  Use chroot system call.
  \item         sys\_ptrace\\
	  Use ptrace to all processes.
  \item         sys\_pacct\\
	  Obtain log of process by acct(2).
	  % allowproc$B$KE}9g$G$-$k!)(B
  \item  sys\_admin\\
	  It grants many operations. 
	  \begin{itemize}
	   \item Usage of following system call: quotactl, mount,
		 umount, swapon, swapoff, sethostname, setdomainname
		 % $B$@$$$?$$@)8f!)(B
	   \item Set attribute to all IPC objects(IPC\_SET) 
		 %IPC$B$GBeMQ!)(B
	   \item Delete all IPC objects(IPC\_RMID)
		 % IPC OWNER$B$G@)8f(B? 
	   \item Set extended security attibute for file system.
		 % setattr$B$H(Brelabel$B$G@)8f(B
	   \item Use fake UID as socket credential.
		 % SETUID,GID$B$GBeMQ(B
	   \item Can open more file than limits in
		 /proc/sys/fs/file-max.
		 % RESOURCE$B$GBeMQ!)(B
	   \item Allocate memory using space reserved for priviledged process.\\
		 It is checked in security\_vm\_enough\_memory LSM hook
		 function. security\_vm\_enough\_memory LSM hook is
		 called in case such as when process is created. The
		 operation is not audited in SELinux.
	   \item Get/set xattr trusted attribute\\
		  Xattr trusted attribute is not used for current SELinux.
	   \item Some ioctl operations\\
		 Developpers of drivers check this capability in some
		 option of ioctl. The check is inserted by developper of
		 driver in place where he thinks important.
	  \end{itemize}
  \item sys\_boot\\
	  Reboot by reboot(2). However, it does not restrict
	  reboot by writing /dev/initctl.
  \item sys\_nice\\
	  Increase nice and change nice for other processes.
	  %$B$+$5$J$k(B
	  
  \item sys\_resource\\
	  \begin{itemize}
	   \item Ignore hardlimit for resource usage in rlimit
	   \item Increase hardlimit for resoure usage in rlimit.
	   \item Use reserved space in ext2 file system
	   \item Modify journal data flag for ext3 by ioctl
	   \item Ignore limit related to message queue in
	  /proc/sys/kernel/msgmnb 	  
	  \end{itemize}
  \item sys\_time\\
	  Modify system clock.
  \item sys\_tty\_config\\
	  Close control terminal by vhangup(2). Change configuration of
	  terminal(such as keycode) by ioctl(such as KDSKBENT, KDSKBSENT option). 
	  % ioctl$B$O(Bhooks.c$B$G;H$o$l$F$k(B
  \item mknod\\
	  Create device file by mknod.
	  %$B=E$J$k(B
  \item lease\\
	  Set lease by fcntl system call. Lease is a kind of lock.When a
	  process sets lease to file, not only file is locked but also
	  signal is sent when other process accesses the file.
	  To use lease, file:lock should also be allowed.

  \item audit\_write\\
	  Send user space AVC message to kernel. User space AVC
	  message is not used in currently SELinux.
  \item audit\_control\\
	  Change configuration of Linux Auditing
	  subsystem(LauS)\cite{laus}
	  %(xxxxhttp://www.uniforum.chi.il.us/slides/HardeningLinux/LAuS-Design.pdf)
	  To change /proc/self/loginuid.
 \end{itemize}
      %%%% POSIX Capability$B$7$i$Y$J$-$c$J!#!#(B
      
\subsection{Object class fd}
\begin{itemize}
 \item use\\
       \begin{itemize}
	\item Inherit file descriptor when process is executed and
	      domain has been changed.
	\item Receive fd from another  process by Unix domain socket\footnote{When creating unix domain
	      socket, by setting SCM\_RIGHTS flag, file descriptor can be sent,
	      see man unix(7).}.
       % selinux_file_receive$B$G%A%'%C%/(B
	\item Get and set attribute of file descriptor,such as owner and flag
	      by fntl and ioctl.
       \end{itemize}       
\end{itemize}
%(3) Unix$B%I%a%$%s%=%1%C%H$G!$B>%W%m%;%9$N%U%!%$%k%G%#%9%/%j%W%?$r<u?.$7(B
%	      $B$?$H$-!#(Bman 7 unix$B$K$h$k$H!$(BSCM_RIGHTS$B$r%;%C%H$9$k$3$H$G!$(B
%	      $B%U%!%$%k<1JL;R$r(BUNIX$B%I%a%$%s%=%1%C%H$G<u?.$G$-$k!#(B

\subsection{Object class filesystem}
SELinux labels superblock of filesystem. permissions in object class
filesystem is used for access control to superblock. 
\begin{itemize}
 \item mount\\
       Mount filesystem.
 \item remount\\
       Remount existing mount by MS\_REMOUNT option of mount(2).
 \item unmount\\
       Unmount filesystem.
 \item getattr\\
       Obtain statistics
       about filesystem, such as free block by statfs(2).
 \item associate\\
       Use type as label for files. A type can not be labeled to file unless the type is not
       associated to file. For example, when we want to use homepage\_t to
       /var/www, and ext3 filesystem is labeled as fs\_t,  then, {\it allow homepage\_t fs\_t filesystem:
       associate; } must be described in policy. 
 \item quotaget\\
       Get quota information .
 \item 	quotamod\\
       Modify quota by quotactl(2).
 \item relabelfrom,relabelto,transition \\
       These are defined in source but are not used.
\end{itemize}

\subsection{Object class process}
permissions in object class are prepared to restrict operations
between process. Unless specified, target type is domain of peer process.
\begin{itemize}
 \item fork\\
       Create new process by fork(2). Target type is the domain itself.
 \item transition\\
       Do domain transition.
 \item sigchld, sigkill, sigstop, signull, signal\\
       Send signal. {\it sigchld} is for SIGCHLD, {\it sigkill} is for SIGKILL,
      {\it  sigstop} is for SIGSTOP and  {\it signull} is for signal number
       zero. {\it signal } is for other signals.
 \item ptrace\\
       Trace process by ptrace(2).

 \item getsched\\
       Read scheduling information of process(such as nice
       value). Session ID is used for job control by shell.
 \item setsched\\
       Modify scheduling information of process.
 \item getsession\\
       Get session ID of process.
       % $B%W%m%;%9$N!V%;%C%7%g%s(BID, $B%W%m%;%9%0%k!<%W(BID$B!W$rF@$k!#$3$l$i$O!$(B
%       http://japan.linux.com/kernel/internal24/node19.shtml$B!$(B
 %      http://www.freebsd.org/doc/ja_JP.eucJP/books/design-44bsd/overview-process-management.html$B!!$"$?$j$,;29M$K$J$k!#(B

 \item getpgid\\
       Get process group ID. Process group ID is used for job control by shell.
 \item	 setpgid\\
       Modify process group ID.
 \item getcap\\
       Get capability information of process by capget(2).
 \item setcap\\
       Modify capability information of process by capset(2).
 \item share\\
       %xxx some pend
       Execute process with domain transition after clone system call. 
 \item getattr\\
       Read process security information of other processes(such as what domain is given) in /proc/pid/attr.
 \item setexec\\
       Set security context of executed process by writing
       /proc/self/attr/exec or by setexecon system call.
 \item setfscreate\\
       Set security context of created file by writing
       /proc/self/attr/fscreate or setfscreatecon system call.
 \item noatsecure\\
       This permission is used for glibc's extended mode(secure mode). When this
       permission is denied, glibc secure mode is enabled(if secure
       mode exists).
 \item siginh\\
       Inherit signal state(such as signal handler) from parent process. This is checked when
       domain has been changed. The default behavior of Linux is to
       inherit signal state(signal handler is not inherited in exec), but by denying this permission, we can
       restrict inheriting signal state. If this is denied, signal state
       is cleared.    

 \item setrlimit\\
       Change rlimit information(resource usage limit) by setrlimit(2).
       
 \item rlimitinh\\
       Inherit rlimit information(resource usage limit information) from
       parent process.This is checked when domain has been changed.  The default behavior of Linux is to
       rlimit information, but by denying this permission, we can
       restrict inheriting rlimit information. If this is denied, rlimit       is cleared.   
 
 \item dyntransition       \\
       Do dynamic domain transition.
       %http://www.nsa.gov/selinux/list-archive/0411/thread_body10.cfm
 \item setcurrent\\
       Set target domain of dynamic domain transition by writing
       /proc/self/current.
 \item execmem, execstack, execheap\\
       These are useful in combination with
       Exec Shield\cite{ExecShield}. These restrict Exec Shield to be
       disabled. For more, see Stephen Smalley's post to SELinux Mailing
       List\footnote{http://marc.theaimsgroup.com/?l=selinux\&m=113440812327410\&w=2}. 
\end{itemize}

	
\subsection{Object class security}
Object class security is operations related to query security
server \footnote {Security Server a component of SELinux which makes
access control decision based on policy },
changing SELinux internal parameters and managing SELinux. The meaning
are found by analyzing selinuxfs.c.

\begin{itemize}
 \item  compute\_av\\
	Query security server about access is denied or granted, by writing /selinux/access.
 \item  compute\_create\\
	Query security server about label transision rule, by writing /selinux/create. 
 \item  compute\_member\\
	Query security server about polyinstantiation\cite{polyinstantation} membership decision, by writing /selinux/member.
	
 \item  check\_context\\
	Query security server about whether security context is valid,
	by writing /selinux/context.
 \item  load\_policy\\
	Load policy file to kernel.
 \item compute\_relabel\\
	Query security server about relabel based on type\_change
	TE rule. {\it type\_change} is a rule to help application to relabel
	object such as tty device.
 \item	compute\_user\\
	Query security server about users that a context can
	reach, by writing /selinux/user. Changing user identity is
	restricted in policy by constraints. 
	This is used programs who change SELinux user identity
	such as login and ssh.
 \item setenforce\\    
	Switch enforcing/permissive mode.
 \item  setbool\\
	Change boolean parameter of policy.
 \item  setsecparam\\
	Configure avc parameter by writing /selinux/avc.
 \item  setcheckreqprot\\
	Configure behavior of permission {\it execmem, execmod and
	execheap} via /selinux/checkreqprot \footnote{By writing 0 or 1
	/selinux/checkreqprot, behavior of execmem, execmod and
       execheap can be configured.}
\end{itemize}

\subsection{Object class system}
In object class system, misc permissions related to system are
defined.

\begin{itemize}
 \item  ipc\_info\\
	Get information about IPC object. This is to get system-wide
	IPC parameter, not information specific to a IPC object. An
	example of system-wide IPC information is segment size of
	shared-memory. More precisely, this controls usage of option
	IPC\_INFO, SHM\_INFO, SEM\_INFO, MSG\_INFO in
	shmctl,semctl,msgctl system call.
 \item 	syslog\_read\\	
	Read kernel message by
	syslog(2)(option 3). 
 \item syslog\_console \\
	Control output of kernel message to console by syslog(2)(option 6,7,8). 
 \item syslog\_mod \\
	Clear kernel message buffer by syslog(2)(option 0,1,2,4,5).
\end{itemize}

\section*{Acknoledgements}
Discussion on NSA's SELinux list, especially Stephen Smalley's
comment was helpful to analyze what access vectors are unused.


\begin{thebibliography}{99}
  \bibitem{seedit} SELinux Policy Editor,
	 URL=http://seedit.sourceforge.net/

 
 
 \bibitem{selinuximplementation} Stephen Smalley, Implementing SELinux
	 as a Linux Secuity Module,
	 URL=http://www.nsa.gov/selinux/papers
 \bibitem{tresysav} An Overview of Object Classes and Permissions,
	 Tresys Technology,
	 URL=http://tresys.com/selinux/obj\_perms\_help.shtml
 \bibitem{lxr} Linux Cross-Reference, URL=http://lxr.linux.no/
 \bibitem{configpolicy} Stephen Smalley, Configuring the SELinux Policy,
	 URL=http://www.nsa.gov/selinux/info/docs.cfm
 \bibitem{ExecShield} Arjan van de Ven, New Security Enhancements in
	 Redhat Enterprize Linux,
	  URL=http://www.redhat.com/f/pdf/rhel/WHP0006US\_Execshield.pdf
 \bibitem{laus} Linux manpage auditd, auditctl, ausearch 
 \bibitem{capp} Controlled Access Protection Profile URL=http://niap.nist.gov/cc-scheme/pp/PP\_CAPP\_V1.d.pdf
 \bibitem{polyinstantation} SELinux Mailing List archive,
	 URL=http://www.nsa.gov/selinux/list-archive/0505/11351.cfm

\end{thebibliography}

\end{document}
