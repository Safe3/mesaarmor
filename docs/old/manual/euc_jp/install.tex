%#!platex manual.tex
\section{INSTALL}


\subsection{Required environment}
\begin{itemize}
\item Distribution 
We tested it work only on Fedora Core4 and Fedora Core3.
Fedora Core 4 is prefered.
\item gcc and libselinux-devel
You need gcc and libselinux-devel.
If you do not have them install by yum, like below.\\
\# yum install gcc\\
\# yum install libselinux-devel\\
\end{itemize}

\subsection{Downloading files}
xxxxxxxxxxxx
\begin{enumerate}
 \item converter
 \item sample policy\\
      
 \item GUI
\end{enumerate}


\subsection{Let's install}
\begin{enumerate}
 \item  Modify run level\\
Example simplified policy does not support X Windows system.
So you have to run system on run level 3.
Modify /etc/inittab like below\\
- Before\\
id:5:initdefault:\\
- After\\
id:3:initdefault:
 \item  install converter\\
\# cd seedit-converter\\
\# make install

 \item install sample policy\\
\# cd seedit-policy
If you are using Fedora Core3, 
edit Make file like below\\
- before: POLICYTYPE=fc4-easy\\
- after: POLICYTYPE=fc3-easy\\
\# make install
 \item reboot\\
At startup, all files are relabeled. It takes some minutes.
After relabel, login as root and reboot again.
\item Switch to enforcing mode\\
Now system is running on permissive mode. You must go to enforcing
	mode. \\
\# setenforce 1\\
And modify /etc/selinux/config\\
- Before\\
SELINUX=permissive\\
- After\\
SELinux=enforcing\\
 \item Enjoy!\\
Minimum system run without configuration. Fore more information, see section
		\ref{sec:spec}.
\end{enumerate}

\subsection{Installing GUI(optional)}
xxxxxx

\subsection{Uninstalling}
Uninstall is quite easy.
\begin{enumerate}
 \item 
\# rm -rf /etc/selinux/seedit-fc4-easy\\
\# rm  /usr/local/bin/converter\\
\# touch /.autorelabel
\item  Modify /etc/selinux/config like below.\\
SELINUX=enforcing\\
SELINUXTYPE=targeted
\end{enumerate}