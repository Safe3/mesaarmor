\documentclass{article}
\usepackage{hyperref}
\title{Specification of Simplified Policy Description Language(SPDL) ver 2.1.a\\(Update for Android).}
\author{An Liu\\liuan03@baidu.com}
\begin{document}
\maketitle
\tableofcontents
\newpage

\section{Overview}
Simplified policy is written in  syntax  called Simplified Policy Description 
Language(SPDL)\footnote{\url{http://seedit.sourceforge.net/doc/2.1/spdl_spec.pdf}}.
 SPDL compiler(seedit-converter) converts SPDL into normal SELinux policy 
language. SPDL has not been updated to support Android since 2009. This 
document describes change/update/addition we made to original SPDL v2.1 during 
our effort of porting Seedit 2.2.0 to Android.

\subsection{Changes to SPDL}
We made following changes to make SPDL compatible with SEAndroid.
\begin{itemize}
	\item SPDL domain grammar change\\
	SEAndroid use attribute to describe a set of domains has common behaviors. 
	We made some changes to SPDL domain grammar and introduce the super domain 
	keyword so that SPDL can reflect the same concept. We also change the 
	domain keyword to solve the conflict with name ``domain''.
	\item Socket grammar change\\
	We modify allownet rule to support option for socket which is not available 
	in SPDL.
	\item Update integrated permissions\\
	There are too many permissions in SELinux, so SPDL reduces number of 
	permissions by removing permissions and integrating permissions. Since 
	latest seedit (2.2.0) supports old Linux Kernel which is not fully 
	compatible with Android's Linux Kernel. We updated some integrated 
	permission set.
	\item Support of SEAndroid features\\
	When SELinux was ported to Android, there are several additional set of 
	permissions added to support unique features of Android, such as: binder, 
	property, Android service, Android KeyStore, DRM(Digital Right Management). 
	We add similar permission sets to SPDL to support Android.
\end{itemize}

\section{SPDL Domain Grammar Change}
\subsection{Original format}
\begin{enumerate}
	\item Syntax:\\
	domain {\em String};
	\item Meaning:\\
	Declare domain. All configuration in a section is done for the declared 
	domain.
	\item Constraints:\\
	Domain name {\em String} must end with {\em \_t}. This is the tradition 
	from SELinux.
\end{enumerate}

\subsection{Updated format}
\begin{enumerate}
	\item Syntax:\\
	{\bf superdomain} {\em String} ;\\
	{\bf domain\_name} {\em String} (: superdomain (, superdomain)*)? ;\\
	\item Meaning:\\
	The first declaration declares superdomain. It is mapped to corresponding 
	attribute in SEAndroid. The second declaration declares a domain which is 
	associated with superdomains listed after ``:''. 
	domain. 
	\item Constraints:\\
	To be compatible with SEAndroid, both domain and superdomain names 
	{\em String} can end without {\em \_t}.
	\item Example:\\
	\begin{verbatim}
		{
		  #corresponding to attribute netdomain in SEAndroid
		  superdomain netdomain;
		  ...
		}
		
		{
		  #declare domain dhcp and associate it with netdomain
		  domain_name dhcp : netdomain;
		  ...
		}
	\end{verbatim}
\end{enumerate}

\section{Socket Grammar Change}

The allownet rule does not support socket class. So we made the following 
change.

\subsection{Updated format}
\begin{enumerate}
	\item Syntax:\\
	allownet -protocol socket create,rw;\\
	allownet -protocol socket -domain name create,rw;\\
	\item Meaning:\\
	The first rule declares that the current domain can create or read/write 
	the socket created by itself. The second rule declares the current domain 
	can create a socket or read/write the socket created by other domain.
	\item Example:\\
	\begin{verbatim}
	{
	  domain_name system_server : netdomain, binderservicedomain, 
	  mlstrustedsubject, bluetoothdomain;
	  ...
	  allownet -protocol socket create;
	}
	
	{
	  domain_name ppp : netdomain;
	  ...
	  allownet -protocol socket -domain mtp create;
	}
	\end{verbatim}
\end{enumerate}

\section{Integrated Permission Update}
Seedit uses spdl\_spec.xml describe integrated SELinux permissions. We updated 
this file to reflect the changes in SEAndroid compared with SELinux. The 
changes can be summarized as follows:
\begin{itemize}
	\item update of SELinux related label from {\em security\_t} to {\em 
	selinuxfs}
	\item add {\em open} permission which is not available in seedit
	\item add additional capabilities (sys\_module, set\_fcap, mac\_override, 
	mac\_admin, syslog, wake\_alarm, block\_suspend, ) for {\em allowpriv}
	\item add {\em getattr} permission for file read operation
	\item additional permissions for Android features (listed in Section 
	\ref{sec:android_change})

\end{itemize}

\section{Native SEAndroid rule inline support}
Some SEAndroid rules cannot be directly translated to SPDL policy. For example, 
Nexus 5 (hammerhead) use a macro {\em qmux\_socket} to do type transition for 
socket file changing label from {\em qmuxd\_socket} to new label combined with 
domain name (e.g., {\em rild\_qmuxd\_socket}). There is no SPDL rule to achieve 
same effect. So we introduce the inline rule so that we can combine SPDL policy 
and SEAndroid/SELinux native policy together.
\begin{enumerate}
	\item Syntax:\\
	inline ``{\em SEAndroid/SELinux native policy}''\\
	\item Meaning:\\
	directly include SEAndroid/SELinux native policy into the final generated 
	sepolicy file
	\item Example:\\
	\begin{verbatim}
	{
	  domain_name mediaserver : netdomain, binderservicedomain, 
	  mlstrustedsubject;
	  ...
	  inline "type_transition mediaserver qmuxd_socket:sock_file 
	  mediaserver_qmuxd_socket;"
	  inline "allow mediaserver mediaserver_qmuxd_socket:sock_file { rename 
	  setattr read lock create getattr write ioctl unlink open append };"
	  ...
	}
	\end{verbatim}
\end{enumerate}

\section{SPDL Support for Android Features}
\label{sec:android_change}
\subsection{Binder}
\begin{enumerate}
	\item Syntax:\\
	allowbinder {\em domain\_list} {\em binder\_permissions};\\
	{\em domain\_list}: list of domains the current domain will call binder 
	operations. Splitted by ``,''.\\
	{\em binder\_permissions}: {\em m, t, c}. splitted by ','.
	\item Meaning:\\
	Allow current domain to initiate binder operations with target domains. 
	{\em m} is for transfer and set\_context\_mgr, {\em t} is for transfer 
	only, 
	and {\em c} is for transfer and call.
	\item Example:\\
	\begin{verbatim}
	{
	  domain_name servicemanager : mlstrustedsubject;
	  domain_trans init /system/bin/servicemanager;
	  ...
	  allowbinder servicemanager m ;
	  allowbinder logd t ;
	}
	
	{
	  domain_name logd : mlstrustedsubject;
	  ...
	  allowbinder su c;
	}
	\end{verbatim}
\end{enumerate}

\subsection{property}
\begin{enumerate}
	\item Syntax:\\
	allowprop {\em property\_name};
	\item Meaning:\\
	The rule allows the current domain to set property named as {\em 
	property\_name}
	\item Example:\\
	\begin{verbatim}
	{
	  
	  allowprop log. ;
	  allowprop security.perf_harden ;
	  allowprop service.adb.root ;
	  allowprop service.adb.tcp.port ;
	  allowprop sys.powerctl ;
	  allowprop sys.usb.ffs. ;
	}
	\end{verbatim}
\end{enumerate}

\subsection{Android service}
\begin{enumerate}
	\item Syntax:\\
	allowsvc {\em service\_name} {\em svc\_permissions};\\
	allowsvc -name {\em attribute\_name} {\em svc\_permissions};\\
	{\em service\_name}: target service name\\
	{\em attribute\_name}: predefined attribute from SEAndroid\\
	{\em svc\_permissions}: {\em a} - add a service, {\em f} - find a service, 
	and {\em l} - list service.
	\item Meaning:\\
	The first rule allow the current domain to perform service related 
	operations to the target service. The second rule allow the current domain 
	to perform service related operations to target services associated with 
	{\em attribute\_name}.
	\item Example:\\
	\begin{verbatim}
	{
	  domain_name untrusted_app : appdomain, netdomain, bluetoothdomain;
	  ...
	  allowsvc -name servicemanager l ;
	}
	
	{
	  domain_name system_server : netdomain, binderservicedomain, 
	  mlstrustedsubject, bluetoothdomain;
	  ...
	  allowsvc -name system_server_service a,f ;
	  allowsvc SurfaceFlinger f ;
	}
	\end{verbatim}
\end{enumerate}

\subsection{Android KeyStore}
\begin{enumerate}
	\item Syntax:\\
	allowks keystore {\em keystore\_permissions};\\
	{\em keystore\_permissions}: {\em i} - insert related permissions (insert, 
	delete, exist, list, get\_state), {\em l} - lock related permissions (lock, 
	unlock, reset, password, is\_empty), {\em s} - sign related permissions 
	(sign, verify), {\em u} - user related permissions (user\_changed, grant, 
	duplicate, clear\_uid), {\em a} - add\_auth permission.
	\item Meaning:\\
	The rule allows the current domain to access keystore with granted 
	permissions.
	\item Example:\\
	\begin{verbatim}
	{
	  domain_name system_server : netdomain, binderservicedomain, 
	  mlstrustedsubject, bluetoothdomain;
	  ...
	  allowks keystore l,i,u,g,s,a ;
	}
	\end{verbatim}
\end{enumerate}

\subsection{DRM}
\begin{enumerate}
	\item Syntax:\\
	allowdrm {\em domain\_name};
	\item Meaning:\\
	This rule allows the current domain to use drm server with {\em 
	domain\_name}.
	\item Example:\\
	\begin{verbatim}
	{
	  domain_name mediaserver : netdomain, binderservicedomain, 
	  mlstrustedsubject;
	  ...
	  allowdrm drmserver ;
	}
	\end{verbatim}
\end{enumerate}


\end{document}
